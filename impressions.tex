\documentclass[oneside,a4paper,twocolumn,platex,dvipdfmx]{jarticle}
\RequirePackage{plautopatch} % パッケージの依存関係を管理
\setlength{\topmargin}{-1.3in}
\setlength{\textheight}{50\baselineskip} % 本文の高さ
\setlength{\textwidth}{47zw} % 本文の幅
\setlength{\footskip}{0pt} % 本文下とフッダ上の距離
\usepackage[dvipdfm]{graphicx} % dvipdfmx を使うため?
\title{達人プログラマーの感想}
\date{2021/02/17}
\author{安彦 久志}

\begin{document}
\maketitle
\thispagestyle{empty}
\section{はじめに}
ソフトウェア開発全般に関わる書籍は初めて読んだ.
研究のために行っていたソフトウェア開発と本書を比較して,
特に気になった内容について述べる.
\section{割れた窓を放置しておかないこと}
完全に放置することはなかった.
対処としては,コメントを残す.
バグが疑われる箇所や修正すべき箇所をリストアップしておく.
などをしていた.
\section{DRY--繰り返しを避けること}
本書では二重化を4つのカテゴリに分割しており,それぞれ思い当たる節があった.
なかでも最も多くやってしまったことは「手抜きによる二重化」であった.
例えば,3回程度の同じ処理であればfor文を使わずに,
記述することがよくあった.
for文を使うよりも,コピー\&ペーストの方がタイプする回数が少ないためだった.
パラメータの設定を2つのプログラムでコードに直書きしていることがあった.
その数値は変更しない前提で開発していたが,
数値に誤りがあり,修正するときに一方のコードしか修正しておらず,
期限ギリギリになってバグが見つかり,
この二重化に気づくのに時間が掛かったことがあった.
\section{モジュール感の結合度を最小にする}
当初は,C++で開発を試みたが,
クラス設計をどうして良いかわからず,
早期にC言語に切り替えた.
しかし,開発が進むと関数に複数の構造体を渡すことが多発した.
また,主要な構造体(顧客,タクシー,道路網)はグローバル変数で定義しており,
煩雑なコードになってしまった.
\section{テスト設計を行うこと}
開発の中でテストは注意深く行っており,
多くのテストをコード内に実装していた.
これにより,多くのバグを発見することができ,かなり役立った.
しかし,全ての機能をテストすることは出来ておらず,
テストコードを組み込む箇所は感覚に頼っていた.
これに対して,本書ではコードの実装より前にテストを作成することを考えている.
期限に追われるなかでテストを先に作成することは,現実的ではないようにも感じた.
しかし,テストを作成することは,ソフトウェアの出力について,
制約条件を確認することになるので,システムを俯瞰して考えることができると感じた.
\section{ドキュメントは付け足すものでなく,組み込むものである}
これまでドキュメントを書いたことはなく,コーディングとは
別の作業だと想像していた.
しかし,コード内に適切なコメントを記述し,
コメントからドキュメントを作成するようなツールを用いることで,
容易にドキュメントを作成できることは効率的だと感じた.
これにより,ドキュメントはコードに対する単なるビューとなるという
考えは,プログラマらしく面白いと思った.

\section{おわりに}
ほどほどに難しくて,応用範囲の広い本は数年読み続けられると感じた.
例えば「柳浦睦憲・茨木俊秀,''組合せ最適化 ―メタ戦略を中心として―'',2001年,朝倉書店」など.
\end{document}
